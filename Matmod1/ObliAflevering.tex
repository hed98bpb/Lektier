\documentclass[paper=a4, fontsize=11pt]{scrartcl} % A4 paper and 11pt font size
\usepackage[utf8]{inputenc}
\usepackage{gensymb}
\usepackage[T1]{fontenc} % Use 8-bit encoding that has 256 glyphs
\usepackage{fourier} % Use the Adobe Utopia font for the document - comment this line to return to the LaTeX default
\usepackage[english]{babel} % English language/hyphenation
\usepackage{amsmath,amsfonts,amsthm} % Math packages

\usepackage{lipsum} % Used for inserting dummy 'Lorem ipsum' text into the template

\usepackage{sectsty} % Allows customizing section commands
\allsectionsfont{\centering \normalfont\scshape} % Make all sections centered, the default font and small caps

\usepackage{fancyhdr} % Custom headers and footers
\pagestyle{fancyplain} % Makes all pages in the document conform to the custom headers and footers
\fancyhead{} % No page header - if you want one, create it in the same way as the footers below
\fancyfoot[L]{} % Empty left footer
\fancyfoot[C]{} % Empty center footer
\fancyfoot[R]{\thepage} % Page numbering for right footer
\renewcommand{\headrulewidth}{0pt} % Remove header underlines
\renewcommand{\footrulewidth}{0pt} % Remove footer underlines
\setlength{\headheight}{13.6pt} % Customize the height of the header

\numberwithin{equation}{section} % Number equations within sections (i.e. 1.1, 1.2, 2.1, 2.2 instead of 1, 2, 3, 4)
\numberwithin{figure}{section} % Number figures within sections (i.e. 1.1, 1.2, 2.1, 2.2 instead of 1, 2, 3, 4)
\numberwithin{table}{section} % Number tables within sections (i.e. 1.1, 1.2, 2.1, 2.2 instead of 1, 2, 3, 4)

\setlength\parindent{0pt} % Removes all indentation from paragraphs - comment this line for an assignment with lots of text

%----------------------------------------------------------------------------------------
%	TITLE SECTION
%----------------------------------------------------------------------------------------

\newcommand{\horrule}[1]{\rule{\linewidth}{#1}} % Create horizontal rule command with 1 argument of height

\title{	
	\normalfont \normalsize 
	\textsc{Aarhus Universitet} \\ [25pt] % Your university, school and/or department name(s)
	\horrule{0.5pt} \\[0.4cm] % Thin top horizontal rule
	\huge Obligatorisk Matematisk Modellering Aflevering \\ % The assignment title
	\horrule{2pt} \\[0.5cm] % Thick bottom horizontal rule
}

\author{Peter Burgaard - 201209175} % Your name

\date{\normalsize\today} % Today's date or a custom date

\begin{document}
	
	\maketitle % Print the title
	
	\begin{enumerate}
		\item Vis, at det kan antages, at variansen er den samme for de syv observationsrækker
	\end{enumerate}
	\begin{align*}
	& M_0: X_{ij} \sim\sim N(\mu_i,\sigma_i^2) \hspace{1cm},\hspace{1cm} i=1, ... , 7\hspace{1cm},\hspace{1cm} j=1,2,3	\\ \\
	& H_0: \sigma_1^2=\sigma_2^2= ... =\sigma_7^2=\sigma^2	\hspace{1cm},\hspace{1cm} \sigma_1^2\leftarrow s^2_{(i)} \sim\sim\sigma_i^2\chi^2(f_{(i)})/f_{(i)} 
	\end{align*}
	Vi laver en bartletts test: \\ 
	Beregning af $-2lnQ(x)$
	\begin{align*}
	 -2lnQ(x) &=f_1lns^2_1-\sum_{i=1}^{k}f_{(i)}lns^2_{(i)}=7*2ln(62.14286)-\sum_{i=1}^{7}f_(i)ln()s^2_{(i)}) \\
	 &=57.8121-56.3413=1.4708\sim\sim\chi^2(k-1) \\
	\end{align*}
	Beregning af c:
	\begin{align*}
	c=1+\frac{1}{3(k-1)}((\sum_{i=1}^{k}\frac{1}{f_{(i)}})-\frac{1}{f_1})=1+\frac{1}{3(7-1)}((\sum_{1}^{7}1/2)-\frac{1}{14})=\frac{25}{21}
	\end{align*}
	Beregning af $Ba$:
	\begin{align*}
	Ba=\frac{-2ln(Q(x))}{c}=\frac{1.4708}{(\frac{25}{21})}=1.235472\sim\sim\chi(k-1)
	\end{align*}
	Beregning af testsandsynlighed
	\begin{align*}
	P_{obs}(x)=1-F_{\chi^2(k-1)}(Ba)=1-\overbrace{\left[ 1\%-2.5\% \right] }^\text{ved opslag}=\left[97.5,99\right]>5\%
	\end{align*}
	Vi forkaster der ikke hypotesen! \\
	Der opstilles ny model
	\begin{align*}
	M_1:X_{ij}\sim\sim N(\mu_i,\sigma2) \hspace{1cm},\hspace{1cm} i=1, ... , 7\hspace{1cm},\hspace{1cm} j=1,2,3
	\end{align*}
	
	\begin{enumerate}
		\item[2.] Indtegn sammenhørende værdier af temperatur og impuls frekvens i et koordinatsystem og vurder ud fra tegningen rimeligheden af en lineær regression af impuls frekvens på temperatur.
	\end{enumerate}
	
	Det optegnede koordinatsystem kan ses i sas-udskriften der er vedlagt som billag, under afsnitet "2) og 3) Optegning af graf og estimation af alpha og beta" \\ 
	
	Der vurderes at punkterne florere omkring en ret linje, og har et bånd af nogenlunde konstant bredde.
	
	
	\begin{enumerate}
		\item[3.] Estimer parametrene i den lineære regression af impuls frekvens på temperatur, indtegn den estimerede regressionslinje i tegningen fra 2 og vis, at det kan antages, at middelværdien af impuls frekvensen afhænger lineært af temperaturen.
	\end{enumerate}
	Vi estimere $\alpha$ og $\beta$
	\begin{align*}
	& SPD_{xt}=SP_{xt}-\frac{S_x*S_t}{n}=153773-\frac{6068*525}{21}=2073 \\
	& SSD_t=USS_t-\frac{S^2_t}{n}=13353-\frac{525^2}{21}=228 \\
	\beta \leftarrow & \hat{\beta}=\frac{SPD_{xt}}{SSD_t}=\frac{2078}{228}=9.0921\sim\sim N(\beta,\frac{\sigma^2}{SSD_t}) \\
	\alpha \leftarrow & \hat{\alpha}=\bar{x.}-\hat{\beta}\bar{t.}=\frac{1}{n}(S_x-\hat{\beta}S_t)=61.6497\sim\sim N(\alpha,\sigma^2(\frac{1}{n})+\frac{\bar{t.}^2}{SSD_t})
	\end{align*}
	Vi opstiller en ny hypotese:
	\begin{align*}
	H_1: &\mu_i=\alpha+\beta t_i
	\end{align*}
	Vi tester.
	Beregning af forskellige værdier til F:
	\begin{align*}
	SSD_{02}&=SSD_x-\frac{SPD_{xt}^2}{SSD_t}&&=1089.02 \\
	SSD_1&=SSD_x?SSD_t&&=870 \\
	SSD_x&=USS_x-\frac{S^2_x}{n}&&=19937 \\
	SSD_t&=USS_t-\frac{S_t^2}{n}&&=228 \\
	SPD_{xt}^2&=SSP_{xt}-\frac{S_x*S_t}{n}=2073^2&&=4298329 \\
	f_{02}&=n-2=7-2&&=5 \\
	f_1&=\sum_{i=1}^{7}n_i-1=7*(3-1)&&=14 \\
	s^2_1&=\frac{\sum_{i=1}^{7}SSD_{(i)}}{\sum_{i=1}^{7}f_{(i)}}=\frac{870}{14}&&=62.142857
	1
	\end{align*}
	Beregning af teststørrelsen F:
	\begin{align*}
	F=\frac{\frac{SSD_{02}-SSD_1}{f_{02}-f_1}}{s^2_1}=\frac{\frac{1089.02-870}{19-14}}{62.1429}=0.704891
	\end{align*}
	Beregning af testsandsynlighed
	\begin{align*}
	P_{obs}=1-F_{F(5,14)}(0.704891)<1-\overbrace{\left[ 0\%-50\% \right] }^\text{ved opslag}>5\%
	\end{align*}
	$H_1$ forkastes ikke, og vi opstiller derfor ny model:
	\begin{align*}
	M_2:X_i\sim\sim N(\alpha+\beta t_i,\sigma^2)\hspace{1cm},\hspace{1cm}i=1, ... ,7
	\end{align*}
	
	\begin{enumerate}
		\item[4.]Undersøg, om det kan antages, at middelværdien af impuls frekvensen forøges med 10, når temperaturen forøges med 1 \degree C.
	\end{enumerate}
	Vi opstiller en ny hypotese:
	\begin{align*}
	H_2:\beta=\beta_0=10& \\ \\t(\underline{x})&=\frac{\hat{\beta}-\beta_0}{\sqrt{\frac{s^2_{02}}{SSD_t}}}=\frac{9.0921-10}{\sqrt{\frac{57.31684}{228}}}=-1.81077\sim\sim t(n-2) \\
	s^2_{02}&=\frac{1}{f_{02}}SSD_{02}=\frac{1}{19}1089.02=57.31684\sim\sim \sigma\chi^2(f_{02})/f_{02}
	\end{align*}
	Beregning af testsandsynligheden:
	\begin{align*}
	P_{obs}(\underline{x})=2\left[1-F_{t(n-2)}(|t(\underline{x})|)\right]=2\left[1-\overbrace{\left[ 95\%-97.5\% \right] }^\text{ved opslag}\right]=\left[5\%,10\%\right]>5\%
	\end{align*}
	Vi forkaster derfor ikke hypotesen, og vi kan opstille en ny model
	\begin{align*}
	& M_3:X_i\sim\sim N(\alpha+\beta_0t_i,\sigma^2)
	\end{align*}
	
\end{document}
