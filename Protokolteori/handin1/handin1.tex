\documentclass[paper=a4, fontsize=11pt]{scrartcl} % A4 paper and 11pt font size
\usepackage[utf8]{inputenc}

\usepackage[T1]{fontenc} % Use 8-bit encoding that has 256 glyphs
\usepackage{fourier} % Use the Adobe Utopia font for the document - comment this line to return to the LaTeX default
\usepackage[english]{babel} % English language/hyphenation
\usepackage{amsmath,amsfonts,amsthm} % Math packages
\usepackage{stmaryrd}

\usepackage{lipsum} % Used for inserting dummy 'Lorem ipsum' text into the template

\usepackage{sectsty} % Allows customizing section commands
\allsectionsfont{\centering \normalfont\scshape} % Make all sections centered, the default font and small caps

\usepackage{fancyhdr} % Custom headers and footers
\pagestyle{fancyplain} % Makes all pages in the document conform to the custom headers and footers
\fancyhead{} % No page header - if you want one, create it in the same way as the footers below
\fancyfoot[L]{} % Empty left footer
\fancyfoot[C]{} % Empty center footer
\fancyfoot[R]{\thepage} % Page numbering for right footer
\renewcommand{\headrulewidth}{0pt} % Remove header underlines
\renewcommand{\footrulewidth}{0pt} % Remove footer underlines
\setlength{\headheight}{13.6pt} % Customize the height of the header

\numberwithin{equation}{section} % Number equations within sections (i.e. 1.1, 1.2, 2.1, 2.2 instead of 1, 2, 3, 4)
\numberwithin{figure}{section} % Number figures within sections (i.e. 1.1, 1.2, 2.1, 2.2 instead of 1, 2, 3, 4)
\numberwithin{table}{section} % Number tables within sections (i.e. 1.1, 1.2, 2.1, 2.2 instead of 1, 2, 3, 4)

\setlength\parindent{0pt} % Removes all indentation from paragraphs - comment this line for an assignment with lots of text

%----------------------------------------------------------------------------------------
%	TITLE SECTION
%----------------------------------------------------------------------------------------

\newcommand{\horrule}[1]{\rule{\linewidth}{#1}} % Create horizontal rule command with 1 argument of height

\title{	
	\normalfont \normalsize 
	\textsc{Aarhus Universitet, Datalogi} \\ [25pt] % Your university, school and/or department name(s)
	\horrule{0.5pt} \\[0.4cm] % Thin top horizontal rule
	\huge Protokolteori - Aflevering 1\\ % The assignment title
	\horrule{2pt} \\[0.5cm] % Thick bottom horizontal rule
}

\author{Peter Burgaard - 201209175} % Your name

\date{\normalsize\today} % Today's date or a custom date

\begin{document}
	
	\maketitle % Print the title
	
	\paragraph{Exercise 1:} Call a function $f:\mathbf{N}\rightarrow\mathbf{R}$ \textit{polynpmial in l} if there exist polynomial $p$ and constant $l_0$ such that $f(l)\leq p(l)$ for all $l>l_0$. Recall that a function $\epsilon:\mathbf{N}\rightarrow\mathbf{R}$ is \textit{negligible in l} if for all polynomials $p$ there exists a constant $l_p$ such that $\epsilon(l)\leq \dfrac{1}{p(l)}$ for all $l>l_p$. \\
	
	\subparagraph{1) Prove that if $\epsilon$ and $\delta$ are negligible in $l$, the $\epsilon + \delta$ is negligible in $l$} \mbox{} \\ \\
	Let $\mathbb{P}[X]$ be all polynomials. \\
	
	If $\epsilon$ and $\delta$ are negligible in $l$ then:
	\begin{align*}
	&\forall p\in\mathbb{P}[X]\forall l>l_p: \epsilon(l)\leq \dfrac{1}{p(l)} \\
	&\forall p'\in\mathbb{P}[X]\forall l'>l_{p'}: \delta(l')\leq \dfrac{1}{p'(l')}
	\intertext{let}
	& l_q=\max\{l_p,l_{p'}\} \\
	\intertext{Since this applies for all polynomials $p\in\mathbb{P}[X]$ defined on $l$, we'll define one as $l^{(c+1)}$, and we have}
	& \forall l>l_q \\
	& \epsilon(l)+\delta(l)\leq 2l^{-(c+1)}\leq l\cdot l^{-(c+1)}=l^{-(c)}=\dfrac{1}{l^c}
	\end{align*}
	Since $l^c\in\mathbb{P}[X]$, and since $\epsilon(l),\delta(l)$ and $\epsilon(l)+\delta(l)\leq\dfrac{1}{l^c}$ we're done. $\hspace{3.5cm}\qed$
	
	\subparagraph{2) Prove that if $\epsilon$ is negligible in $l$ and $f$ is polynomial in $l$, the $f\cdot \epsilon$ is negligible in $l$} \mbox{} \\
	
	Assume that there exists a $f\in\mathbb{P}[X]$ such that $\epsilon(l)\cdot f(l)\not\leq\dfrac{1}{p(l)}\forall p\in\mathbb{P}[X]$, this would imply 
	$$\epsilon(l)\not\leq\dfrac{1}{\left(\dfrac{p(l)}{f(l)}\right)}=\dfrac{1}{h(l)}$$
	which again would mean $h(l)\not\in\mathbb{P}[X]$. Since $\epsilon(l)$ is negligible $\forall p\in\mathbb{P}[X]\forall l>l_p \hspace{1.5cm} \lightning$
	Therefore $\epsilon(l)\cdot f(l)\leq\dfrac{1}{p(l)}$ and is negligible. $\hspace{7cm}\qed$
	
\end{document}
