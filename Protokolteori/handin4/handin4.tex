\documentclass[paper=a4, fontsize=11pt]{scrartcl} % A4 paper and 11pt font size
\usepackage[utf8]{inputenc}

\usepackage[T1]{fontenc} % Use 8-bit encoding that has 256 glyphs
\usepackage{fourier} % Use the Adobe Utopia font for the document - comment this line to return to the LaTeX default
\usepackage[english]{babel} % English language/hyphenation
\usepackage{amsmath,amsfonts,amsthm} % Math packages

\usepackage{lipsum} % Used for inserting dummy 'Lorem ipsum' text into the template

\usepackage{sectsty} % Allows customizing section commands
\allsectionsfont{\centering \normalfont\scshape} % Make all sections centered, the default font and small caps

\usepackage{fancyhdr} % Custom headers and footers
\pagestyle{fancyplain} % Makes all pages in the document conform to the custom headers and footers
\fancyhead{} % No page header - if you want one, create it in the same way as the footers below
\fancyfoot[L]{} % Empty left footer
\fancyfoot[C]{} % Empty center footer
\fancyfoot[R]{\thepage} % Page numbering for right footer
\renewcommand{\headrulewidth}{0pt} % Remove header underlines
\renewcommand{\footrulewidth}{0pt} % Remove footer underlines
\setlength{\headheight}{13.6pt} % Customize the height of the header

\numberwithin{equation}{section} % Number equations within sections (i.e. 1.1, 1.2, 2.1, 2.2 instead of 1, 2, 3, 4)
\numberwithin{figure}{section} % Number figures within sections (i.e. 1.1, 1.2, 2.1, 2.2 instead of 1, 2, 3, 4)
\numberwithin{table}{section} % Number tables within sections (i.e. 1.1, 1.2, 2.1, 2.2 instead of 1, 2, 3, 4)

\setlength\parindent{0pt} % Removes all indentation from paragraphs - comment this line for an assignment with lots of text

%----------------------------------------------------------------------------------------
%	TITLE SECTION
%----------------------------------------------------------------------------------------

\newcommand{\horrule}[1]{\rule{\linewidth}{#1}} % Create horizontal rule command with 1 argument of height

\title{	
	\normalfont \normalsize 
	\textsc{Aarhus University - Computer Science} \\ [25pt] % Your university, school and/or department name(s)
	\horrule{0.5pt} \\[0.4cm] % Thin top horizontal rule
	\huge Protocol Theory - Handin 4 \\ % The assignment title
	\horrule{2pt} \\[0.5cm] % Thick bottom horizontal rule
}

\author{Peter Burgaard -201209175} % Your name

\date{\normalsize\today} % Today's date or a custom date

\begin{document}
	
	\maketitle % Print the title
	
	\section*{Exercise 4}
	
	Consider the unconditionally hiding commitment scheme based on discrete logarithms, where the public key is $pk=(p,g,y)$ for a prim $p$, a generate $g$ of $\mathbb{Z}_p^*$ and $y\in \mathbb{Z}_p^*$. And a commitment to $b$ using randomness $r$ has form $commit_{pk}(r,b)=y^bg^r\mod{p}$. The randomness $r$ is chosen uniformly from $\mathbf{Z}_{p-1}={0,1,2,\cdots,p-2}$.
	Suppose a prover $P$ has committed to bits $b_1,b_2$ using commitments $c_1,c_2$ where $b_1\not=b_2$. Now $P$ wants to convince the verifier $V$ that the bits are different. We claim he can do this by sending to $V$ a number $s\in\mathbb{Z}_{p-1}$ such that $c_1c_2=yg^s\mod{p}$
	\begin{itemize}
		\item Show how an honest $P$ can compute the requiced $s$, and argue that the distribution of $s$ is the same when $(b_1,b_2)=(0,1)$ as when $(b_1,b_2)=(1,0)$. This means that $V$ learns nothing excepts that $b_1\not=b_2$
		
		No matter the combination of $(b_1,b_2)$ it is seen that $$(y^1g^{r})\cdot(y^0g^{r'})=yg^{r+r'}=(y^0g^{r})\cdot(y^1g^{r'})$$
		It is thus posible for $P$ to compute $s=r+r'\mod{p-1}$
		
		\item Argue that if $P$ has in fact committed in $c_1,c_2$ to $(0,0)$ or $(1,1)$, he cannot efficiently find $s$ as above unless he can compute the discrete logarithm of $y$.
		
		We can calculate $s$ as follows
		
		\begin{align*}
		yg^s=& (y^{b_1}g^{r})\cdot(y^{b_2}g^{r'})=y^{b_1+b_2}g^{r+r'}  \intertext{from this we can isolate}
		g^s=& y^{(b_1+b_2)-1}g^{r+r'} \implies
		g^{s-(r-r')}= y^{(b_1+b_2)-1} \intertext{which means by taking the log, we get}
		&\log_g(y^{(b_1+b_2)-1})+r+r'=s
		\end{align*}
		
		\item Argue in a similar way that $P$ can convince $V$ that he has committed to two bits that are \textit{equal} by revealing $s$ such that $c_1c_2^{-1}=g^s\mod{p}$
		
		We see, that we will always have $g^s$ with $s=r-r'\mod{p-1}$ since 
		\begin{align*}
		(y^{b_1}g^{r})\cdot(y^{b_2}g^{r'})^{-1}&=(y^{b_1}g^{r})\cdot(y^{-b_2}g^{-r'}) \\
		&=y^{b_1-b_2}g^{r-r'} \\
		&=g^{r-r'} \\
		&=g^s
		\end{align*}
		
	\end{itemize}
    

    
\end{document}

