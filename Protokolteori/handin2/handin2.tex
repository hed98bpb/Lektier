\documentclass[paper=a4, fontsize=11pt]{scrartcl} % A4 paper and 11pt font size
\usepackage[utf8]{inputenc}

\usepackage[T1]{fontenc} % Use 8-bit encoding that has 256 glyphs
\usepackage{fourier} % Use the Adobe Utopia font for the document - comment this line to return to the LaTeX default
\usepackage[english]{babel} % English language/hyphenation
\usepackage{amsmath,amsfonts,amsthm} % Math packages

\usepackage{lipsum} % Used for inserting dummy 'Lorem ipsum' text into the template

\usepackage{sectsty} % Allows customizing section commands
\allsectionsfont{\centering \normalfont\scshape} % Make all sections centered, the default font and small caps

\usepackage{fancyhdr} % Custom headers and footers
\pagestyle{fancyplain} % Makes all pages in the document conform to the custom headers and footers
\fancyhead{} % No page header - if you want one, create it in the same way as the footers below
\fancyfoot[L]{} % Empty left footer
\fancyfoot[C]{} % Empty center footer
\fancyfoot[R]{\thepage} % Page numbering for right footer
\renewcommand{\headrulewidth}{0pt} % Remove header underlines
\renewcommand{\footrulewidth}{0pt} % Remove footer underlines
\setlength{\headheight}{13.6pt} % Customize the height of the header

\numberwithin{equation}{section} % Number equations within sections (i.e. 1.1, 1.2, 2.1, 2.2 instead of 1, 2, 3, 4)
\numberwithin{figure}{section} % Number figures within sections (i.e. 1.1, 1.2, 2.1, 2.2 instead of 1, 2, 3, 4)
\numberwithin{table}{section} % Number tables within sections (i.e. 1.1, 1.2, 2.1, 2.2 instead of 1, 2, 3, 4)

\setlength\parindent{0pt} % Removes all indentation from paragraphs - comment this line for an assignment with lots of text

%----------------------------------------------------------------------------------------
%	TITLE SECTION
%----------------------------------------------------------------------------------------

\newcommand{\horrule}[1]{\rule{\linewidth}{#1}} % Create horizontal rule command with 1 argument of height

\title{	
	\normalfont \normalsize 
	\textsc{Aarhus University - Computer Science} \\ [25pt] % Your university, school and/or department name(s)
	\horrule{0.5pt} \\[0.4cm] % Thin top horizontal rule
	\huge Protocol Theory - Handin 2 \\ % The assignment title
	\horrule{2pt} \\[0.5cm] % Thick bottom horizontal rule
}

\author{Peter Burgaard -201209175} % Your name

\date{\normalsize\today} % Today's date or a custom date

\begin{document}
	
	\maketitle % Print the title
	
	\section*{Exercise 6}
	
	Let $p$ be a prime, an $g$ an elemen in the multiplicative group $\mathbb{Z}_p^*$ of order $q$. Such an element always exist if $q$ divide $p-1$. \\ \\
	Note that the function $f:\mathbb{Z}_{p-1}\rightarrow \mathbb{Z}_p^*, x\mapsto g^x\mod{p}$ is a group homomorphism, namely $f(x+y\mod{p-1})=f(x)f(y)\mod{p}$. It is generally believed that $f$ is one-way for large enough primes $p,q$. \\ \\
	Use this assumption to construct an unconditionally hiding and computationally binding bit commitment scheme - use the RSA construction as inspiration, and write down the commitment function explicitly. Argue that if one can efficiently break the binding property, one can also efficiently solve the discrete log problem $\mod{p}$, i.e., invert $f$. Note that it is part of the construction that one can check that public key is correctly formed, in particular $y$ must be in $Im(f)$. For this you may use (without proof) the fact that in $\mathbb{Z}_p^*$, an element is in the subgroup genereted by $g$ if and only if it has order $q$. \\
	
	We generate a public $pk=f(x)=g^x\mod{q}=y$ for $x\in_R\mathbb{Z}_{p-1}$ and $g\in_R\mathbb{Z}_p^*$. When commiting we use the function $commit_{pk}(b,r)=y^b\cdot f(r)\mod{q}=(g^x)^b\cdot g^r\mod{q}=g^{x\cdot b+r}\mod{q}$. Lets set $c=x\cdot b+r$, such that $commit_{pk}(b,r)=g^c\mod{q}$. If the prover were to change his mind, he would have to find a number $k$ such that $g^x\cdot f(k)\mod{q}=g^{x\cdot k}\mod{q}=g^c\mod{q}$. Finding such a $k$ is the same as solving the Diffie-Hellman problem, which by Lemma 1 in [DLBS - Damgård] is no harder than the discrete log problem. This implies, for a prover the binding property only holds if he can't solve the discret log problem. \\ \\
	
	As a matter of checking whether a $pk$ is valid, since we map to a group of order $q$ where each element is a generator of the whole group, so the verifier may send $q$ such that we can calculate $pk^q\mod{q}=pk$ and $f(r)^q\mod{q}=f(r)$ for some $r\in_R\mathbb{Z}_{p-1}$, just to verify that the $pk$ isn't a special element with the wanted characteristic. This should insure that $pk\in Im(f)$.
	

\end{document}
