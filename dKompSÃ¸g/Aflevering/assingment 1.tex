\documentclass[paper=a4, fontsize=11pt]{scrartcl} % A4 paper and 11pt font size
\usepackage[utf8]{inputenc}

\usepackage[T1]{fontenc} % Use 8-bit encoding that has 256 glyphs
\usepackage{fourier} % Use the Adobe Utopia font for the document - comment this line to return to the LaTeX default
\usepackage[english]{babel} % English language/hyphenation
\usepackage{amsmath,amsfonts,amsthm} % Math packages

\usepackage{lipsum} % Used for inserting dummy 'Lorem ipsum' text into the template

\usepackage{sectsty} % Allows customizing section commands
\allsectionsfont{\centering \normalfont\scshape} % Make all sections centered, the default font and small caps

\usepackage{fancyhdr} % Custom headers and footers
\pagestyle{fancyplain} % Makes all pages in the document conform to the custom headers and footers
\fancyhead{} % No page header - if you want one, create it in the same way as the footers below
\fancyfoot[L]{} % Empty left footer
\fancyfoot[C]{} % Empty center footer
\fancyfoot[R]{\thepage} % Page numbering for right footer
\renewcommand{\headrulewidth}{0pt} % Remove header underlines
\renewcommand{\footrulewidth}{0pt} % Remove footer underlines
\setlength{\headheight}{13.6pt} % Customize the height of the header

\numberwithin{equation}{section} % Number equations within sections (i.e. 1.1, 1.2, 2.1, 2.2 instead of 1, 2, 3, 4)
\numberwithin{figure}{section} % Number figures within sections (i.e. 1.1, 1.2, 2.1, 2.2 instead of 1, 2, 3, 4)
\numberwithin{table}{section} % Number tables within sections (i.e. 1.1, 1.2, 2.1, 2.2 instead of 1, 2, 3, 4)

\setlength\parindent{0pt} % Removes all indentation from paragraphs - comment this line for an assignment with lots of text

%----------------------------------------------------------------------------------------
%	TITLE SECTION
%----------------------------------------------------------------------------------------

\newcommand{\horrule}[1]{\rule{\linewidth}{#1}} % Create horizontal rule command with 1 argument of height

\title{	
	\normalfont \normalsize 
	\textsc{Aarhus University, Computer Science} \\ [25pt] % Your university, school and/or department name(s)
	\horrule{0.5pt} \\[0.4cm] % Thin top horizontal rule
	\huge dKomp Assignment 1 \\ % The assignment title
	\horrule{2pt} \\[0.5cm] % Thick bottom horizontal rule
}

\author{Peter Burgaard - 201209175} % Your name

\date{\normalsize\today} % Today's date or a custom date

\begin{document}
	
	\maketitle % Print the title
	
	\section{Proof of proposition 1}
	
	If $\pi_1$ and $\pi_2$ are good representations that are polynomially equicalent, then $L_1\in \textbf{P} \iff L_2\in \textbf{P}$. \\
	
	We will only show $\Rightarrow$ part, because the $\Leftarrow$ is equivalent. 
	
	Let  
	\begin{align*}
	&\pi_i:S\rightarrow\{0,1\}^* \hspace{2cm} \text{for } i=1,2  \\
	& L_1=\{x|f(\pi_1^{-1}(x))=yes\} \\
	& L_2=\{x|f(\pi_2^{-1}(x))=yes\}
	\end{align*}
	
	$\pi_1$ being a good representation means $\pi_1(S)\in \textbf{P}$, i.e., it can be decided efficiently if a given string is a valid representation of an object.
	$\pi_1$ and $\pi_2$ being polynomial equivalent implies $\exists r_1, r_2$ translations between $\pi_i$, i.e., $\forall x\in S$, $\pi_1(x)=r_1(\pi_2(x))$ and $\pi_2(x)=r_2(\pi_1(x))$. \\
	
	So $\pi_1$ being a good representation and the existens of and $r_1$, due to $\pi_1$ and $\pi_2$ being polynomial equivalent, means that we can translate any $\pi_2(x)\forall x\in S$ to an $\pi_1(x)$ representation, which can be decided in polynomial time. Which means $L_1\in\textbf{P}\Rightarrow L_2\in\text{P}$. \\
	
	The same argument can be applied to get $\Leftarrow$.
	\qed

\end{document}
