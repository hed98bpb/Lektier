\documentclass[paper=a4, fontsize=11pt]{scrartcl} % A4 paper and 11pt font size
\usepackage[utf8]{inputenc}

\usepackage[T1]{fontenc} % Use 8-bit encoding that has 256 glyphs
\usepackage{fourier} % Use the Adobe Utopia font for the document - comment this line to return to the LaTeX default
\usepackage[english]{babel} % English language/hyphenation
\usepackage{amsmath,amsfonts,amsthm} % Math packages

\usepackage{lipsum} % Used for inserting dummy 'Lorem ipsum' text into the template

\usepackage{sectsty} % Allows customizing section commands
\allsectionsfont{\centering \normalfont\scshape} % Make all sections centered, the default font and small caps

\usepackage{fancyhdr} % Custom headers and footers
\pagestyle{fancyplain} % Makes all pages in the document conform to the custom headers and footers
\fancyhead{} % No page header - if you want one, create it in the same way as the footers below
\fancyfoot[L]{} % Empty left footer
\fancyfoot[C]{} % Empty center footer
\fancyfoot[R]{\thepage} % Page numbering for right footer
\renewcommand{\headrulewidth}{0pt} % Remove header underlines
\renewcommand{\footrulewidth}{0pt} % Remove footer underlines
\setlength{\headheight}{13.6pt} % Customize the height of the header

\numberwithin{equation}{section} % Number equations within sections (i.e. 1.1, 1.2, 2.1, 2.2 instead of 1, 2, 3, 4)
\numberwithin{figure}{section} % Number figures within sections (i.e. 1.1, 1.2, 2.1, 2.2 instead of 1, 2, 3, 4)
\numberwithin{table}{section} % Number tables within sections (i.e. 1.1, 1.2, 2.1, 2.2 instead of 1, 2, 3, 4)

\setlength\parindent{0pt} % Removes all indentation from paragraphs - comment this line for an assignment with lots of text

%----------------------------------------------------------------------------------------
%	TITLE SECTION
%----------------------------------------------------------------------------------------

\newcommand{\horrule}[1]{\rule{\linewidth}{#1}} % Create horizontal rule command with 1 argument of height

\title{	
	\normalfont \normalsize 
	\textsc{Aarhus University, Science, Datalogi} \\ [25pt] % Your university, school and/or department name(s)
	\horrule{0.5pt} \\[0.4cm] % Thin top horizontal rule
	\huge Optimering - Compulsory assignment 3 \\ % The assignment title
	\horrule{2pt} \\[0.5cm] % Thick bottom horizontal rule
}

\author{Peter Burgaard, 201304189\\
	\large Lavet i samarbejde med Kresten Maigaard Axelsen 2013, Niels Bross, 201304189} % Your name

\date{\normalsize\today} % Today's date or a custom date

\begin{document}
	
	\maketitle % Print the title
	
	\section{Problem D}
	
	There are $n$ indivisible items of weights $w_1,\dots ,w_n$ to be distributed in $m$ bags.  We want to minimize the weight of the heaviest bag. Show how to formulate this as an integer linear program.  Explicitly state the set of variables, the objective function, the constraints and explain the meaning of each variable and eachc onstraint. How  should  your  program  be  changed  if  we  instead  want  to  maximize  the  weight  of  thelightest bag? \\	\\
	Den overordnede ide er denne løsning er, find den tungeste taske, og minimer dens vægt. Vi konstruere følgende variable
	
	\begin{align*}
	&x_{ij} = \begin{cases}
	1 \text{ hvis taske $i$ vælge til vægt $j$} \\
	0 \text{ ellers}
	\end{cases}
	\end{align*}
	
	
	Vi bruger et matematisk max, for at finde den tungeste taske
	
	\begin{align*}
	&\max_{i=1}^{m} \sum_{j=1}^{n} w_j x_{ij} = k
	\end{align*}
	
	dernæst minimeres der over denne max taske k
	
	\begin{align*}
	&\min k\\
	&\text{s.t.}\\
	&\sum_{j=1}^n x_{ij} = 1, i = 1,2,\cdots,m\\ \intertext{Denne constraint sørger for, at hver vægt bruges præcist én gang på taske} 
	&\sum_{i=1} w_j x_{ij} \leq k, j = 1,\cdots,n \intertext{Søger for, at vi ikke får en taske der er tungere end den allerede tungst fundne taske}
	& x_{ij} \in \{0,1\}\\ \intertext{Enten er en vægt med i tasken, eller også er den ikke med, dermed har vi denne binæree værdi, som er defineret i dens gaffel function}
	\end{align*} 
	
	Som vist overfor er vores objective function $k$.
	
	\begin{itemize}
		\item How  should  your  program  be  changed  if  we  instead  want  to  maximize  the  weight  of  the lightest bag?
	\end{itemize}
	
	Det syntes oplagt, nærmest blot at følge det ovenstående. Så vi maximere vægten af den letteste taske først i stedet
	
	\begin{align*}
	&\max_{i=1}^{m} \sum_{j=1}^{n} w_j x_{ij} = k
	\end{align*}
	
	herefter kan vi nærmest genbruge overstående
	
	\begin{align*}
	&\max k\\
	&\text{s.t.}\\
	&\sum_{j=1}^n x_{ij} = 1, i = 1,2,\cdots,m\\
	&\sum_{i=1} w_j x_{ij} \geq k, j = 1,\cdots,n \\
	&x_{ij} \in \{0,1\}\\
	\end{align*} 
	
	hvor vores constraints egentlig står for det samme, på nær et ulighedstegn der er vendt om for, at den letteste taske ikke bliver lettere end den allerede fundne tungeste af de lette tasker, 
	
	
\end{document}

