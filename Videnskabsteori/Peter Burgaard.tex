\documentclass[paper=a4, fontsize=12pt]{scrartcl} % A4 paper and 11pt font size
\usepackage{enumitem}
\usepackage[T1]{fontenc} % Use 8-bit encoding that has 256 glyphs
\usepackage[utf8]{inputenc}

\usepackage[english]{babel} % English language/hyphenation
\usepackage{amsmath,amsfonts,amsthm} % Math packages
\usepackage{pslatex}
\usepackage{lipsum} % Used for inserting dummy 'Lorem ipsum' text into the template

\usepackage{sectsty} % Allows customizing section commands
\allsectionsfont{\centering \normalfont\scshape} % Make all sections centered, the default font and small caps

\usepackage{fancyhdr} % Custom headers and footers
\pagestyle{fancyplain} % Makes all pages in the document conform to the custom headers and footers
\fancyhead{} % No page header - if you want one, create it in the same way as the footers below
\fancyfoot[L]{} % Empty left footer
\fancyfoot[C]{} % Empty center footer
\fancyfoot[R]{\thepage} % Page numbering for right footer
\renewcommand{\headrulewidth}{0pt} % Remove header underlines
\renewcommand{\footrulewidth}{0pt} % Remove footer underlines
\setlength{\headheight}{13.6pt} % Customize the height of the header

\numberwithin{equation}{section} % Number equations within sections (i.e. 1.1, 1.2, 2.1, 2.2 instead of 1, 2, 3, 4)
\numberwithin{figure}{section} % Number figures within sections (i.e. 1.1, 1.2, 2.1, 2.2 instead of 1, 2, 3, 4)
\numberwithin{table}{section} % Number tables within sections (i.e. 1.1, 1.2, 2.1, 2.2 instead of 1, 2, 3, 4)

\setlength\parindent{0pt} % Removes all indentation from paragraphs - comment this line for an assignment with lots of text

\linespread{1.5}

%----------------------------------------------------------------------------------------
%	TITLE SECTION
%----------------------------------------------------------------------------------------

\newcommand{\horrule}[1]{\rule{\linewidth}{#1}} % Create horizontal rule command with 1 argument of height

\title{	
	\normalfont \normalsize 
	\textsc{Aarhus university - Science Studies/Videnskabsteori} \\ [25pt] % Your university, school and/or department name(s)
	\horrule{0.5pt} \\[0.4cm] % Thin top horizontal rule
	\huge Informationsteknologiens og datalogiens videnskabsteori og etik \\ % The assignment title
	\horrule{2pt} \\[0.5cm] % Thick bottom horizontal rule
}

\author{Peter Burgaard -20120175} % Your name

\date{\normalsize\today} % Today's date or a custom date

\begin{document}
	
	\maketitle % Print the title
	
	
	\paragraph{(1 - a.}
	
	According to Karl Popper, there must be two things present within a scientific theory. First and foremost, theories are made to describe behavior of some aspects of the world or universe\cite{POP61}. Therefore Popper argues  if the claim in question does not rule out some possible scenarios the claim cannot explain anything, and therefore does not qualify for the status of theory\cite{POP59}. Secondly the claim must be falsifiable. This implies  "... if there exists a logically possible observation statement or set of observation statements that are inconsistent with it, that is, which, if established as true, would falsify the hypothesis."\cite{POP62}. 
	
	\paragraph{(1 - b.}
	
	Since "... the falsity of universal statements can be deduced from suitable singular statements"\cite{POP61}, the falsifiability of the claim "sometimes it rains from morning till evening", is determined by the presence of a possible scenario which contradicts it. In this case we have three contradicting scenarios:
	\begin{enumerate}[nolistsep]
		\item It has never rained
		\item It has always rained
		\item It has rained, but never the entirety of the period 
	\end{enumerate}  
	All three can be ruled out because they're not applicable for the whole world at once, which implies the claim is not falsifiable. If the claim had been location specific, scenario three might be applicable, and therefor make it falsifiable.
	
	\paragraph{(2 - a.}
	
	Thomas Kuhn argues "... history of science should see science as developing onwards from a given point in history."\cite{KUHN20}. These given points in time are what Kuhn describes as paradigm shifts. He explains these as given points in the timeline of the development of a science, which is as follows, paradigm --> crisis --> revolution --> new paradigm, also called a paradigm shift\cite{KUHN20}. \\
	One of Kuhn’s descriptions of a paradigm is "... the entire constellation of beliefs, values, techniques, ... shared by the members of a given community "\cite{KUHN23}. It is from these paradigms scientists find the tools and techniques to conduct normal science\cite{KUHN21}. Within the normal science we may encounter anomalies, which are unsolvable problems given the established paradigms tools and exemplars\cite{KUHN25}.\\
    Kuhn argues, when severe anomalies occur we enter a state of crisis and the scientific community moves from normal science to "extraordinary research"\cite{KUHN27}. This is a state where the rules of normal science becomes prograssively blurred, in the effort of trying to solve the anomalies\cite{KUHN28}. 
	According to Kuhn there are different ways the science may continue at the end of a crisis\cite{KUHN29}, one of them being, a solution is found by the extraordinary research, and these ideas will either replace or be incorporated into the existing paradigm, which means a paradigm shift\cite{KUHN29}.

	\paragraph{(2 - b.}

	MCK\footnote{Martin Campbell-Kelly (MCK)} and WA\footnote{William Aspray (WA)} uses the paradigm concept to describe different understandings and definitions of three commercial computers, which were launched during 1977\cite{MCKWA220}. MCK \& WA claims, most people had an idea about the physical shaping of a computer, but lacked an idea of usage an general purposes\cite{MCKWA221}. That is why when Commodore, Tandy and Apple, created three computers, for different target groups with different purposes in mind, they created three different ways of viewing these computers usage, and therefore three definitions of what a computer could be. \\
	MCK \& WA's usage of the concept paradigm, varies somewhat from Kuhn's. The greatest difference is found in the area of usage, where Kuhn uses paradigms in a field of science to define some general consensus within the field. MCK \& WA uses it with no reference to science at all, but as a point in history where something is defined by its usage, or at least the idea for its usage. This implies, the definition Kuhn uses, depends upon the integration of the understanding and consensus into a given community, before it is a paradigm, where as MCK \& WA's usage can be just one specific company's " ... existing culture and corporate outlook."\cite{MCKWA220} being integrated into a product, which should give rise to a new paradigm. It seems MCK \& WA uses paradigm to define when someone claims to have found, in this case the personal computers, usage, and not upon the general consensus of the particular item usage and intergation into the targeted users, as is the case with Kuhn. 

	\paragraph{(3 - a.}
	
	The LIM\footnote{Linear Innovation Model (LIM)}, by Vannevar Bush\footnote{Or at least by most, credited for making it, even though there is no real reference to it in his book 'Science: the endless frontier'\cite{GODIN4}}, is a model " ... that suggests technical change happens in a linear fashion from Invention to Innovation ... "\cite{WIKILMOI}. This implies innovation starts with basic research(science without a specific scope or goal) from which new knowledge is created, which might result in inventions. Next phase is (technical) development, in which the idea or invention is adjusted, and improved until reaching a functioning state, where it is usable and sellable. Finally we move into the last phase which is innovation, where the product is successfully introduced to the market, which relates it to being accepted by society and profitable.

	\paragraph{(3 - b.}
	
	WITSP\footnote{'Who Invented the Smart Phone?' (WITSP)}, describes how the first smartphone was IBM's Simon\cite{OMI1}, which was released in mid 1994\cite{WIKIIBM}. By the LIM, the release was to soon due to the fact IBM didn't manage to make Simon sellable. WITSP then continues to introduce Nokia’s communicator, which was the first smartphone with color touchscreen, as the next evolution of the smartphone\cite{OMI2}\footnote{This, on the otherhand isn't true, due to the nokia 7710 being the first nokia phone with a touch and color display. \cite{NOKIA1}\cite{NOKIA2}\cite{NOKIA3}}. The reason this didn't catch on either may have been due to the phone network being centerred around speach, and not having the data speeds which could make full use of some of the smartphones capabilities\cite{OMI1}\cite{NYT1}. The iPhone, which was launched in 2007, "... marked the transition of smartphones being devices for business people to devices that the average consumber might desire"\cite{OMI3}. The LIM shows that, Apple were the first to incorperate the right technical development, due to the end product being both usable and sellable, which made it an innovation, being both acceptable in the society and highly profitable for Apple.
	
	\paragraph{(3 - c.}
	
	I would argue the LIM presents a dynamic market too simplisticly in the respect of releasing new products, and how well they will succeed. The lack of success for IBM's Simon and Nokia's communicator outlined in WITSP\cite{OMI1}\cite{OMI2} is not solely because of the phones them selfs, but the surrounding services, in which the phones are used with. If we look past the lack of data speed in the early 2000 and before\cite{NYT1}, it can be argued the internet was not as widely used in the "pre-socialmedia-age", and certainly not at the same frequency as it is today. With the number of internet users in Europe and north America increasing more that 500 \% the last 15 years\cite{IWS}, and the number of internet users more than tripling, since Nokia's 7710 smartphone, world wide\cite{ILS}, the demand for internet on one's phone would not be so highly prioritized, as might have been thought by the "pre-iPhone-creators". \\
	The LIM does not involve iteration, which I would argue is one of the biggest reasons for the succes and innovation phones have had. Running through the list of Nokia phones being made, there are alot of phones with many different designs and secondary functionalities (beside voice calls)\cite{NOKIA1}. The information Nokia would gain from these many different releases, would have helped them in the design of later releases.
	
	\paragraph{(4 - a.}
	
	Hacker ethic is a term attributed to Journalist Teven Levy as described in his 1984 book titled "Hackers: Heroes of the Computer Revolution"\cite{HEM}. Levy defines hacker ethic " ... as a project undertaken or a product built to fulfill some constructive goal, but also with some wild pleasure taken in mere involvement."\cite{HEM}. More general tenets of hacker ethic, includes: Sharing, all information should be free, openness, "Criteria such as age, sex, race, position, and qualification are deemed irrelevant within the hacker community ... "\cite{HE1}, decentralization, all bureaucracy is bad, and overall world improvement. \\ 
	In chapter two in Levy[1984], Levy goes on to define his 'hands-on imperative': "Hackers believe that essential lessons can be learned about the systems - about the world - from taking things apart, seeing how they work, and using this knowledge to create new and more interesting things."\cite{HE12}. Due to the fact that the hands-on imperative requires free access to information and knowledge, Wikipedia claims, if a true hacker is met with restrictions, then the ends would justifies the means, " ... so that improvements can be made ... "\cite{HE12}. It is in such statements, a thin line between criminal and idealist can be drawn, and where the hackers true ethics will show.
	
	\paragraph{(4 - b.}
	
	It could be argued that allocating more researchers to a given research area could be beneficial for  society, being the area are chosen with society in mind. It can benefit society by delivering the much needed solution faster, but also weaken the given field of research the newly allocated researchers are allocated from. Given what seems like an ever expanding need for computer scientists, with a job growth of 20 \% through 2020\cite{CW}, future allocation of research power to any given project, can arguably in the future cause greater damage to non allocated research areas, due to a job market outpacing the number of workers to manage them. \\
	If society chose another route by not allocating scientists to any given areas, and letting them choose freely, which project to participate in it could be argued that by having chosen freely the given researcher might produce greater results working in a specific field which holds a special interest for that given researcher. This might lead to less beneficial result for society if the scientific field has little or no real world application when the research is conducted. But by doing this, society might on the other hand not run the risk of draining specific research areas, and having an evenly balanced research population across all sciences.
	
	\paragraph{(5 -} I somewhat agree with John Searle. The very definition of what thinking is, is a difficult concept to define. On one hand, one could argue that all thinking is, is input through ones life which generates an output through different ideas and impressions left by the world around one. This should be possible transform into to a function, which would make it almost impossible for humans to differentiate AI from real human intelligence. I argue that maybe it is not within the human capacity to identify what thinking actually means. Just because one thinks, one might not be able to identify how oneself is doing it, or even what it is oneself is doing. Like a child prodigy in singing, might be able to hear the notes from its voice, and account for the melody they are singing, they may not know what they are doing with their vocal cords, or even what they mentally do to produce such sounds. As so humans might be able to tell what they are thinking, and what might have inspired them to reach that thought, but to identify what thinking is, is a whole different matter. This is why I somewhat reach the same conclusion as Searle, but in a different way. \\
	I will argue that the Turing test is unfair, in the sense that I do not see the argument which should justify why a human is applicable to the task of identifying when something is thinking. If we follow the premise of the Turing test, and the Chinese room, I argue that there must logically exist a reverse Turing test, in which a machine with a implemented computable function should be able to decide the ability to think of its test subject. And a reverse test is conducted, is it sufficient for the reverse test to conclude another computer can think, for it to be sufficiently correct? I think Searle is right when he argues that everything done by a computer is nothing but a simulation. Since a computer only answers through predefined functions, these functions will still be a simulation of what thinking beings are doing, and this does not make the computer a thinking entity.
	
	
	\newpage
	
	\begin{huge}
		Literature overview
	\end{huge}
	
	\begin{itemize}
		\item A. F. Chalmers, 'What is this thing called Science?' third edition
		\item H. Andersen, 'Wadsworth Philosophers series on Kuhn'
		\item M. Campbell-Kelly \& W. Aspray, 'Computer: A history of the information machine', Oxford: Westview Press, 2004
		\item B. Godin, 'The Linear Model of Innovation: The Historical Construction of an Analytical Framework', Science, Technology, \& Human Values, Vol. 31, No. 6 (Nov., 2006)
		\item http://en.wikipedia.org/wiki/Linear\_model\_of\_innovation
		\item https://www.ownmyinvention.com/who-invented-the-smart-phone/
		\item http://en.wikipedia.org/wiki/IBM\_Simon
		\item http://en.wikipedia.org/wiki/Nokia\_7710
		\item http://www.gsmarena.com/nokia\_7710-review-31.php
		\item http://www.theverge.com/2013/9/3/4689034/nokia-iconic-mobile-phone-photo-essay
		\item http://www.nytimes.com/2004/09/23/technology/23verizon.html?8br\&\_r=0
		\item http://www.internetworldstats.com/stats.htm
		\item http://www.internetlivestats.com/internet-users/
		\item http://en.wikipedia.org/wiki/Hacker\_ethic
		\item http://www.computerworld.com/article/2502348/it-management/it-jobs-will-grow-22--through-2020--says-u-s-.html
	\end{itemize}
	
	\newpage
	
	\begin{thebibliography}{99}
		
		\bibitem{POP61}
		A. F. Chalmers, 'What is this thing called Science?' third edition, chapter 5 'Introducing falsification', 'Falsifiability as a criterion for theories', p. 61.
		
		\bibitem{POP59}
		A. F. Chalmers, 'What is this thing called Science?' third edition, chapter 5 'Introducing falsification', 'Introducition', p. 59.
		
		\bibitem{POP62}
		A. F. Chalmers, 'What is this thing called Science?' third edition, chapter 5 'Introducing falsification', 'Falsifiability as a criterion for theories', p. 62.
		
		\bibitem{KUHN20}
		H. Andersen, 'Wadsworth Philosophers series on Kuhn', chapter 3 'The structure of Scientific Revolutions', 'Main concepts of Structure', p. 20.
		
		\bibitem{KUHN23}
		H. Andersen, 'Wadsworth Philosophers series on Kuhn', chapter 3 'The structure of Scientific Revolutions', 'Paradigm', p. 23.
		
		\bibitem{KUHN21}
		H. Andersen, 'Wadsworth Philosophers series on Kuhn', chapter 3 'The structure of Scientific Revolutions', 'Normal science', p. 20.
		
		\bibitem{KUHN25}
		H. Andersen, 'Wadsworth Philosophers series on Kuhn', chapter 3 'The structure of Scientific Revolutions', 'Anomalies', p. 25.
		
		\bibitem{KUHN27}
		H. Andersen, 'Wadsworth Philosophers series on Kuhn', chapter 3 'The structure of Scientific Revolutions', 'Crisis and extraordinary science', p. 27.
		
		\bibitem{KUHN28}
		H. Andersen, 'Wadsworth Philosophers series on Kuhn', chapter 3 'The structure of Scientific Revolutions', 'Crisis and extraordinary science', p. 28.
		
		\bibitem{KUHN29}
		H. Andersen, 'Wadsworth Philosophers series on Kuhn', chapter 3 'The structure of Scientific Revolutions', 'Revolution', p. 29.
		
		\bibitem{MCKWA220}
		M. Campbell-Kelly \& W. Aspray, 'Computer: A history of the information machine', Oxford: Westview Press, 2004, 'Shaping of the personal Computer', p. 220
		
		\bibitem{MCKWA221}
		M. Campbell-Kelly \& W. Aspray, 'Computer: A history of the information machine', Oxford: Westview Press, 2004, 'Shaping of the personal Computer', p. 221'
		
		\bibitem{GODIN4}
		B. Godin, 'The Linear Model of Innovation: The Historical Construction of an Analytical Framework', Science, Technology, \& Human Values, Vol. 31, No. 6 (Nov., 2006), pp. 639-667, p. 5
		
		\bibitem{WIKILMOI}
		Wikipedia - Linear model of innovationhttp://en.wikipedia.org/wiki/Linear\_model\_of\_innovation, last accessed March 22, 2015
		
		\bibitem{OMI1}
		https://www.ownmyinvention.com/who-invented-the-smart-phone/, 'Simon: The First Smartphone', published August 1, 2014, last accessed March 22, 2015
		
		\bibitem{OMI2}
		https://www.ownmyinvention.com/who-invented-the-smart-phone/, 'Other Early Phones', published August 1, 2014, last accessed March 22, 2015
		
		\bibitem{WIKIIBM}
		Wikipedia - IBM Simon, http://en.wikipedia.org/wiki/IBM\_Simon, last accessed March 22, 2015
		
		\bibitem{NOKIA1}
		http://www.theverge.com/2013/9/3/4689034/nokia-iconic-mobile-phone-photo-essay, last accessed March 22, 2015
		
		\bibitem{NOKIA2}
		Wikipedia - Nokia 7710, http://en.wikipedia.org/wiki/Nokia\_7710, last accessed March 22, 2015
		
		\bibitem{NOKIA3}
		http://www.gsmarena.com/nokia\_7710-review-31.php, last accessed March 22, 2015
		
		\bibitem{NYT1}
		M. Ritchel, 'Verizon Wireless Expands High-Speed Data Network', http://www.nytimes.com/2004/09/23/technology/23verizon.html?8br\&\_r=0, published September 23, 2004, last accessed March 22, 2015
		
		\bibitem{OMI3}
		https://www.ownmyinvention.com/who-invented-the-smart-phone/, 'The iPhone', published August 1, 2014, last accessed March 22, 2015
		
		\bibitem{IWS}
		http://www.internetworldstats.com/stats.htm, last accessed March 22, 2015
		
		\bibitem{ILS}
		http://www.internetlivestats.com/internet-users/, last accessed March 22, 2015
		
		\bibitem{HEM}
		Wikipedia - Hacker Ethic Intro. http://en.wikipedia.org/wiki/Hacker\_ethic last accessed March 22, 2015
		
		\bibitem{HE1}
		Wikipedia - Hacker Ethic 1 - The hacker ethics. http://en.wikipedia.org/wiki/Hacker\_ethic last accessed March 22, 2015
		
		\bibitem{HE12}
		Wikipedia - Hacker Ethic 1.2 - Hands-On Imperative, http://en.wikipedia.org/wiki/Hacker\_ethic last accessed March 22 2015
		
		\bibitem{CW}
		http://www.computerworld.com/article/2502348/it-management/it-jobs-will-grow-22--through-2020--says-u-s-.html, last accessed March 22 2015
	\end{thebibliography}

\end{document}
