\documentclass[paper=a4, fontsize=11pt]{scrartcl} % A4 paper and 11pt font size

\usepackage[T1]{fontenc} % Use 8-bit encoding that has 256 glyphs
\usepackage{fourier} % Use the Adobe Utopia font for the document - comment this line to return to the LaTeX default
\usepackage[english]{babel} % English language/hyphenation
\usepackage{amsmath,amsfonts,amsthm} % Math packages

\usepackage{lipsum} % Used for inserting dummy 'Lorem ipsum' text into the template

\usepackage{sectsty} % Allows customizing section commands
\allsectionsfont{\centering \normalfont\scshape} % Make all sections centered, the default font and small caps

\usepackage{fancyhdr} % Custom headers and footers
\pagestyle{fancyplain} % Makes all pages in the document conform to the custom headers and footers
\fancyhead{} % No page header - if you want one, create it in the same way as the footers below
\fancyfoot[L]{} % Empty left footer
\fancyfoot[C]{} % Empty center footer
\fancyfoot[R]{\thepage} % Page numbering for right footer
\renewcommand{\headrulewidth}{0pt} % Remove header underlines
\renewcommand{\footrulewidth}{0pt} % Remove footer underlines
\setlength{\headheight}{13.6pt} % Customize the height of the header

\numberwithin{equation}{section} % Number equations within sections (i.e. 1.1, 1.2, 2.1, 2.2 instead of 1, 2, 3, 4)
\numberwithin{figure}{section} % Number figures within sections (i.e. 1.1, 1.2, 2.1, 2.2 instead of 1, 2, 3, 4)
\numberwithin{table}{section} % Number tables within sections (i.e. 1.1, 1.2, 2.1, 2.2 instead of 1, 2, 3, 4)

\setlength\parindent{0pt} % Removes all indentation from paragraphs - comment this line for an assignment with lots of text

%----------------------------------------------------------------------------------------
%	TITLE SECTION
%----------------------------------------------------------------------------------------

\newcommand{\horrule}[1]{\rule{\linewidth}{#1}} % Create horizontal rule command with 1 argument of height

\title{	
	\normalfont \normalsize 
	\textsc{Aarhus Univeristy} \\ [25pt] % Your university, school and/or department name(s)
	\horrule{0.5pt} \\[0.4cm] % Thin top horizontal rule
	\huge Science Studies - handin 1 \\ % The assignment title
	\horrule{2pt} \\[0.5cm] % Thick bottom horizontal rule
}

\author{Peter Burgaard - 201209175} % Your name

\date{\normalsize\today} % Today's date or a custom date

\begin{document}
	
	\maketitle % Print the title
	
	\section{Summery of Kuhns philosophy of science, based on Andersen 2001}
	
	Kuhn argues that "instead of describing the cumulative development towards a specific point in history, the present, history of science should see science as developing onwards from a given point in history."\footnote{"Kuhn", Andersen 2001, page 20"}. These given points in time is what Kuhn describes as paradigmshifts. He explain these as point in a scientific development which is a follows, paradigm --> crisis --> revolution --> new paradigm, or a paradigm shift. Between these paradigm shifts normal science is done. This Kuhn defines as "research based firmly upon one or more past scientific achievements, archievements that some particular scientific community acknowledges for a time as supplying the foundations for its succes"\footnote{"Kuhn", Andersen 2001, page 21}. Within this normal science, Kuhn argues, the aim isn't at "calling forth new sort of phenormena or at inventing new theories"\footnote{"Kuhn" Andersen 2001, page 21}, but "increasing the succes of the accepted theory"\footnote{"Kuhn" Andersen 2001, page 21}. He further devides normal science into three classes: determination of significat facts, mathinc of facts with theory or articulation of theory. Because this is done within the rules of the sciences paradigm, the tools, for which the normal science is cunducted, are standardized, allows the science to move faster and deeper, because there's a consensus about what's right and wrong. This focuses the scientist in their research. Kuhn, in his postscript of the secound edition of "The structures of Scientific Revolutions", defines paradigms in two different senses. "On the one hand, it stands for the entire constellation of beliefs, values, techniques, and so on shared by the members of a given cumminity. On the other, it denotes one sort of element in that constellation, the concrete puzzle-solutions which, employed as models or examples, can replace explicit rules as a basis for the solution of the remaining puzzles of normal science"\footnote{"Kuhn", Andersen 2001, page 23}. To show this constellation Kuhn creates a disciplinary matrix, which contains "... the symbolic generalizations, that is, the scientific laws in their most fundamental forms ... beliefs about which objects and phenomena exist in the world, ... values by which the quality of research can be evaluated, ... and exemplary problems and problem silutions which he now called exmaplars rather than paradigms. ... exemplars (are) over explicit definition-like rules."\footnote{"Kuhn", Andersen 2001, page 23}. Kuhn argues that when these exemplars come short in explaining a newly discovered phenomena or problem within the given field, there'll be a problem. These problem he defines as Anomalies. He further describes these anomalies as "the recognition that nature has somehow viiolated the paradigm-induced expectations that govern normal sciecne"\footnote{"Kuhn", Andersen 2001, page 25}. He goes on that defining a severe kind of anomalies, which "... leads to questioning the accepted tools and understandings ..."\footnote{"Kuhn", Andersen 2001, page 26} within the science. When these severe anomalies are discovered crisis within the science may appear. When this happens Kuhn argues that normal-science transitions to extraordinary science. In this transition more and more of the science community within the fields is dedicated to the problem, until either it is solved, or the so many scientists are working on it that it may become \textit{the} matter of the subject. He explains this extraordinary science, as a loosening of the rules of normal science were more and more "ad hoc." science is conducted. When/if a solution is found, a new paradigm is startin to be put into place which Kuhn calls the revolution. He argues that this transition within the field is incremental for the community but not for the individual scientist. Kuhn thinks that either you follow or believe the old paradigm, og change your ways and start following the rules within the new paradigm. When this new paradigm is the only paradigm followed within the field, science starts anew, and will not reach a new crisis/if at all, untill it reaches a level of maturity.
	
	
	\section{Summery of scientific development, based on groupe text}
	
	In my groupe text, "The fourth paradigm", Gordon Bell's main point is revolving about what he calls the fourth paradigm. He argues that the three previous paradigm science has gone through has been; empirical - generel decribtions of natural phenomena, theoretical - the use of simulation and generalization, computational - the use of simulation of complex phenomena. Bell argues that we're on the verge of a new paradigm where the use of data gathering instrument will reach new heights, due to the fact of data mining and generel data management being developed in such a way that previousely, what might have been seen as, infinetly large amount of data, can be automatically sorted and categorized, so new scientific discoveries can be made. Bell moves on to claim that the times where a persons whole scientific life is recorded and put online is not that far ahead. He argues that previously enourmes amount of data and scientific documentation, have been lost due to unorganized personal notes and documentation, in which only the scientist him/herself could navigate through. When the scientist died or moved on to different project, the data may have been kept, but rarely is it ever categorized, or sorted for evaluation of scientific material the scientist him/herself might have overlooked, or didn't have the necessary tools for evaluatiing. 
	\\ \\
	He moves on to talk about how the unification of scientific readings being put on the internet leads to a need for a new way to peer-review ones findings. He gives an example of a new paper being made, where people can come with suggestions about which paper could of use for this articles. The creators of the new articles will read them, and give them mention and credits in the new article. The database management system will then keep track of highly used papers, and therefor will the frequently recieve higher rankings.
	
	\section{Disussion}
	
	The use of paradigm in Gordon Bell's presentation isn't used as Kuhn defines it. Bell uses a paradigm merly as a way to define how we gather information within science. As mentioned, this doesn't match neither of the definitions Kuhn gives in his postscripts. It can be argued that given the normal science is conducted through the paradigms, and the paradigm lays the rules of concensus within the science for how we achieving scientific result, that by changing the ways we use tools to get to our conclussion we are also chainging out paradigm. This, I believe will only result in a paradigmshift if and only if, these new tools can deliver new results within a specific field which was impossible before the use of data mining/data base management system. 
	\\ \\
	Kuhn mentions that the use of paradigms gives science a focus and standardization which is very beneficial because, it allows different scientists to understand each other, and directly use each others result in the process of making normal science. This could further be the case in the use of massive online libraries. Bell mentions that by using database systems, not only for ones papers, but also for data gathered through different experiments, scientists can gaing an even deeper understanding for each others results, by directly reviewing each other findings and even recreate them, to see if it maches ones own findings. 
	\\ \\
	Bell uses alot of his presentation, much less claiming this to be a scientific revolution, which will change science as we know it. Since Kuhn would ague that only through some sort of severe anomaly, will we reach a state where a true revolution of the science can take place, Bell seems to think, that by giving scientists these new tools, the use of already existing tools would change. Being able to analyse in greater depth ones enormous data sets, could change the way science is being conducted. But to Kuhn this wouldn't matter much. It might well change some of the basic practices, but the exemplars from which on creates experiments wouldn't change. The results from these new data sets might well only reconfirm what we already know. Just as Kuhn argues, when anomalies doesn't quite exist within the world the scientist operates in, then how is the use of new categorization of data going to show new results? It seems like the we practices would remain the same, and even the instruments would remain the same.
	\\ \\
	The use of online libraries, and upload of ones scientific life on the otherhand could well lead to new discoveries. Maybe by having multiple scientists reaching same anomaly conclution for a specific experiment, but disclaiming for different reasons. But this is all speculation. The result one might reach, might have significance for the individual sciences but, maybe only within the normal science in which they are roaming.  
	
	\section{Definition of paradigm}

	In newer times the term paradigm has change more and more to revolve around paradigm within small new branches of sciences. Many of the methods used within the science may be very similar to the science the branch originated from but new methods are need to explore this new field. An example is nursing, where alot of articles with the word paradigm in their header or body has been produced in the late 1990's. In that sence a real paradigm shifting discovery might have been found, because many of the models or tools used from the science they originated from might not be suffiecient so they are in a state of extraordinare science, as Kuhn would put it. To other more mature sciences where real paradigm shifting discoveries only comes mayber once or twice every centery this might seem excessive, but maybe it isn't.
	\\ \\
	Recently paradigm has also been used more and more as a sort of buzzword, for self advertisement of ones paper. Jon Cohen mentions a scientist Daniel Steinberg in his article "The March of Paradigm" who commented; "I though we should reserve 'new paradigm' for Darwin, Freud and Newton"\footnote{"The March of Paradigms", Cohen, 1999, page 2}. If this opinion is shared with larger parts of the scientific community, then a word such as paradigm should surely turn a head or two when appering in the title of a paper.   

\end{document}
