\documentclass[paper=a4, fontsize=11pt]{scrartcl} % A4 paper and 11pt font size
\usepackage[utf8]{inputenc}

\usepackage[T1]{fontenc} % Use 8-bit encoding that has 256 glyphs
\usepackage{fourier} % Use the Adobe Utopia font for the document - comment this line to return to the LaTeX default
\usepackage[english]{babel} % English language/hyphenation
\usepackage{amsmath,amsfonts,amsthm} % Math packages
\usepackage{subcaption}
\usepackage{graphicx}
\usepackage{lipsum} % Used for inserting dummy 'Lorem ipsum' text into the template

\usepackage{sectsty} % Allows customizing section commands
\allsectionsfont{\centering \normalfont\scshape} % Make all sections centered, the default font and small caps

\usepackage{fancyhdr} % Custom headers and footers
\pagestyle{fancyplain} % Makes all pages in the document conform to the custom headers and footers
\fancyhead{} % No page header - if you want one, create it in the same way as the footers below
\fancyfoot[L]{} % Empty left footer
\fancyfoot[C]{} % Empty center footer
\fancyfoot[R]{\thepage} % Page numbering for right footer
\renewcommand{\headrulewidth}{0pt} % Remove header underlines
\renewcommand{\footrulewidth}{0pt} % Remove footer underlines
\setlength{\headheight}{13.6pt} % Customize the height of the header

\numberwithin{equation}{section} % Number equations within sections (i.e. 1.1, 1.2, 2.1, 2.2 instead of 1, 2, 3, 4)
\numberwithin{figure}{section} % Number figures within sections (i.e. 1.1, 1.2, 2.1, 2.2 instead of 1, 2, 3, 4)
\numberwithin{table}{section} % Number tables within sections (i.e. 1.1, 1.2, 2.1, 2.2 instead of 1, 2, 3, 4)

\setlength\parindent{0pt} % Removes all indentation from paragraphs - comment this line for an assignment with lots of text

%----------------------------------------------------------------------------------------
%	TITLE SECTION
%----------------------------------------------------------------------------------------


%----------------------------------------------------------------------------------------
%	TITLE SECTION
%----------------------------------------------------------------------------------------

\newcommand{\horrule}[1]{\rule{\linewidth}{#1}} % Create horizontal rule command with 1 argument of height

\title{	
	\normalfont \normalsize 
	\textsc{Aarhus Universitet, Science, Computer Science} \\ [25pt] % Your university, school and/or department name(s)
	\horrule{0.5pt} \\[0.4cm] % Thin top horizontal rule
	\huge Cryptography - Handin 5 \\ % The assignment title
	\horrule{2pt} \\[0.5cm] % Thick bottom horizontal rule
}

\author{Peter Burgaard - 201209175} % Your name

\date{\normalsize\today} % Today's date or a custom date

\begin{document}

\maketitle

\section{Exercise 4}

Suppose we are given a cryptosystem $(G,E,D)$. Assume an adversary develops an algorithm $Alg$ running in time $T$ that can take a ciphertext $E_K(x)$ form an m-bit plaintext x and compute the first bit of x. 

\paragraph{Describe an adversary that plays the game in the CPA security definition and uses $Alg$ to try to distinguish the real and ideal case.}  \hspace{1cm} \\

Since K is fixed for the entire attack, for our adversary to distinguish the real from the ideal case, he could feed the oracle the same input, and run the $Alg$ repeatedly on each output, and if an output from $Alg$ comes out with a different first bit than the others, the adversary would know he's in the ideal case due, to the fact that it would be a random string r being encrypted, and not a fixed message x. Of cause the chance of each new r having the same first bit after iteration n, would be $2^{-n}$ and by this method, the adversary would not be guaranteed a stop point in his effort to correctly figure out which case he's in. 

if same = real, og not = ideal

\paragraph{Which advantage can you obtain?} \hspace{1cm} \\

Since we can check when dealing with the real oracle if the output is correctly encrypted, and there's a $50\%$ chance of the ideal oracle given us a message with the correct starting bit, our advantage would be
\begin{align*}
|1-0.5|=0.5
\end{align*}

\paragraph{In terms of the parameters $(t,q,\mu, \epsilon)$, which parameter values does your adversary obtain?} \hspace{1cm} \\

Since our algorithm take time T to run, $T=t$. We only need on call to make our decision which implies $q=1$. Since the text string we send to the oracle is of length m, $\mu=m$. And since we have advantage 0.5, this means $\epsilon\geq0.5$.

\paragraph{How would your result change if $Alg$ cannot compute the first bit with certainty but only guess it with probability $p>\dfrac{1}{2}$?} \hspace{1cm} \\



P(R,R)= P
P(R,I)=0,5p+(1*p)*0.5=-0.5
Adc(r,I)=p-0.5

\end{document}