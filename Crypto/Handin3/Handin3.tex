\documentclass[paper=a4, fontsize=11pt]{scrartcl} % A4 paper and 11pt font size
\usepackage[utf8]{inputenc}

\usepackage[T1]{fontenc} % Use 8-bit encoding that has 256 glyphs
\usepackage{fourier} % Use the Adobe Utopia font for the document - comment this line to return to the LaTeX default
\usepackage[english]{babel} % English language/hyphenation
\usepackage{amsmath,amsfonts,amsthm} % Math packages

\usepackage{lipsum} % Used for inserting dummy 'Lorem ipsum' text into the template

\usepackage{sectsty} % Allows customizing section commands
\allsectionsfont{\centering \normalfont\scshape} % Make all sections centered, the default font and small caps

\usepackage{fancyhdr} % Custom headers and footers
\pagestyle{fancyplain} % Makes all pages in the document conform to the custom headers and footers
\fancyhead{} % No page header - if you want one, create it in the same way as the footers below
\fancyfoot[L]{} % Empty left footer
\fancyfoot[C]{} % Empty center footer
\fancyfoot[R]{\thepage} % Page numbering for right footer
\renewcommand{\headrulewidth}{0pt} % Remove header underlines
\renewcommand{\footrulewidth}{0pt} % Remove footer underlines
\setlength{\headheight}{13.6pt} % Customize the height of the header

\numberwithin{equation}{section} % Number equations within sections (i.e. 1.1, 1.2, 2.1, 2.2 instead of 1, 2, 3, 4)
\numberwithin{figure}{section} % Number figures within sections (i.e. 1.1, 1.2, 2.1, 2.2 instead of 1, 2, 3, 4)
\numberwithin{table}{section} % Number tables within sections (i.e. 1.1, 1.2, 2.1, 2.2 instead of 1, 2, 3, 4)

\setlength\parindent{0pt} % Removes all indentation from paragraphs - comment this line for an assignment with lots of text

%----------------------------------------------------------------------------------------
%	TITLE SECTION
%----------------------------------------------------------------------------------------

\newcommand{\horrule}[1]{\rule{\linewidth}{#1}} % Create horizontal rule command with 1 argument of height

\title{	
	\normalfont \normalsize 
	\textsc{Aarhus University - Computer Science department} \\ [25pt] % Your university, school and/or department name(s)
	\horrule{0.5pt} \\[0.4cm] % Thin top horizontal rule
	\huge Crypthology - Handin 3 \\ % The assignment title
	\horrule{2pt} \\[0.5cm] % Thick bottom horizontal rule
}

\author{Peter Burgaard - 201209175} % Your name

\date{\normalsize\today} % Today's date or a custom date

\begin{document}
	
	\maketitle % Print the title
	
	\section{Crypto Systems - Problem 2}
	
	\section{Stinson - 2.12}
	Prove that, in any cryptosystem $H(K|C)\geq H(P|C)$ \\ \\
	Basicly we have to prove $H(K|C) - H(P|C) \geq 0 \implies H(K|C)\geq H(P|C)$. The proof is as follows
	\begin{proof}
	\begin{align*}
	\intertext{By Stinson theorem 2.10}
	H(K|C) - H(P|C) &= H(K) + H(P) - H(C) - H(P|C)
	\intertext{From the proof section of 2.10 have}
	H(K,P,C) = H(K,P) &= H(K) + H(P)
	\intertext{Which implies}
	H(K) + H(P) - H(C) - H(P|C) &= H(K,P,C) - H(C) - H(P|C)
	\intertext{By theorem 2.8}  
	H(K,P,C) - H(C) - H(P|C) &= H(K,P,C) -H(P,C)
	\intertext{Again by theorem 2.8}
	H(K,P,C) -H(P,C) &= H(K|P,C)
	\intertext{Since entropy can never be negativ, we get}
	H(K|C) - H(P|C) &= H(K|P,C) \geq 0
	\end{align*}
	\end{proof}
	\section{Stinson - 2.14}
	Compute $H(K|C)$ and $H(K|P,C)$ for the \textit{Affine Cipher}, assuming that keys are used equiprobably and the plaintext are equiprobable. \\ \\
	Since $|\mathcal{P}|=26$, and the letters are equiprobably chosen, we get from Stinson p. 55 
	\begin{align*}
	H(P)=log_2(|\mathcal{P}|)=log_2(26)\approx 4.7
	\end{align*}
	Since a key $K$ is a pair $(a,b)$, where $a,b\in\mathbb{Z}$ and $gcd(a,26)=1$, we have 26 different b's and $\phi(26)=12$ a's. This implies 312 different, keypairs, which, again, are equiprobably chosen:
	\begin{align*}
	H(K)=log_2(|\mathcal{K}|)=log_2(312)\approx 8.285
	\end{align*}
	Becuase the plain text and the key are equiprobably chosen, and affine uniquely encodes every $x\in\mathcal{P}$ to $e_K(x)=y\in\mathcal{C}$ we see that, the probability of $y$ is the same as $x$, which implies:
	\begin{align*}
	H(C)=H(P)\approx4.7
	\end{align*}
	This gives us by theorem 2.10, Stinson:
	\begin{align*}
	H(K|C) = H(K) + H(P) - H(C) = H(K) \approx 8.285
	\end{align*}
	By exercise 2.12 Stinson, and theorem 2.8 
	\begin{align*}
	H(K|P,C) = H(K|C) - H(P|C) \approx 8.285 - 4.7  \approx 3.584\\
	\end{align*}
	
	H(K|P)-H(P|C)=H(K)+H)P)-H(P)-H(C)
	
\end{document}
