\documentclass[paper=a4, fontsize=11pt]{scrartcl} % A4 paper and 11pt font size
\usepackage[utf8]{inputenc}
\usepackage[T1]{fontenc} % Use 8-bit encoding that has 256 glyphs
\usepackage{fourier} % Use the Adobe Utopia font for the document - comment this line to return to the LaTeX default
\usepackage[english]{babel} % English language/hyphenation
\usepackage{amsmath,amsfonts,amsthm} % Math packages

\usepackage{lipsum} % Used for inserting dummy 'Lorem ipsum' text into the template
\usepackage{marvosym}
\usepackage{sectsty} % Allows customizing section commands
\allsectionsfont{\centering \normalfont\scshape} % Make all sections centered, the default font and small caps

\usepackage{fancyhdr} % Custom headers and footers
\pagestyle{fancyplain} % Makes all pages in the document conform to the custom headers and footers
\fancyhead{} % No page header - if you want one, create it in the same way as the footers below
\fancyfoot[L]{} % Empty left footer
\fancyfoot[C]{} % Empty center footer
\fancyfoot[R]{\thepage} % Page numbering for right footer
\renewcommand{\headrulewidth}{0pt} % Remove header underlines
\renewcommand{\footrulewidth}{0pt} % Remove footer underlines
\setlength{\headheight}{13.6pt} % Customize the height of the header

\numberwithin{equation}{section} % Number equations within sections (i.e. 1.1, 1.2, 2.1, 2.2 instead of 1, 2, 3, 4)
\numberwithin{figure}{section} % Number figures within sections (i.e. 1.1, 1.2, 2.1, 2.2 instead of 1, 2, 3, 4)
\numberwithin{table}{section} % Number tables within sections (i.e. 1.1, 1.2, 2.1, 2.2 instead of 1, 2, 3, 4)

\setlength\parindent{0pt} % Removes all indentation from paragraphs - comment this line for an assignment with lots of text

%----------------------------------------------------------------------------------------
%	TITLE SECTION
%----------------------------------------------------------------------------------------

\newcommand{\horrule}[1]{\rule{\linewidth}{#1}} % Create horizontal rule command with 1 argument of height

\title{	
	\normalfont \normalsize 
	\textsc{Aarhus University - Computer Science department} \\ [25pt] % Your university, school and/or department name(s)
	\horrule{0.5pt} \\[0.4cm] % Thin top horizontal rule
	\huge Crypthology - Handin 9 \\ % The assignment title
	\horrule{2pt} \\[0.5cm] % Thick bottom horizontal rule
}

\author{Peter Burgaard - 201209175} % Your name

\date{\normalsize\today} % Today's date or a custom date

\begin{document}
	
	\maketitle % Print the title

\section*{If the DDH problem is hard (w.r.t. GGen), then the El Gamal cryptosystem is CPA secure}
\begin{proof}
We will prove this by proof of contradiction. Assume the existence of an adversary \textit{Adv} with advantage greater than $\epsilon$ against the El Gamal cryptosystem. We will make a polynomial time reduction DDH$\leq$El Gamal. 
\\ \\ 
The reduction is as follows; \\
The DDH is built on a group G, generator $\alpha$ and constants $a$ and $b$ which are used to generate $\alpha^a$, $\alpha^b$ and $\alpha^{a\cdot b}$. Given a DDH cryptosystem we built our El gamal cryptosystem as such:
\begin{enumerate}
\item Base the system on the same group as DDH, lets name it G'
\item Base the system on the same generator as DDH, lets name it $\alpha'$
\item Set $\beta:=\alpha'^a$, note that DDH's $\alpha^a=\alpha'^a$ in our El Gamal cryptosystem
\item Encoding a messages is then based on the pair ($\alpha^b$, $\alpha'^{a\cdot b}\cdot m$)
\end{enumerate}
It is easily varified that none of these steps takes more than polynomial time. 
\\ \\
By construction \textit{Adv} would take ($\alpha^b$, $\alpha'^{a\cdot b}\cdot m$) and be able to decide whether he received the encrypted chosen message back from the oracle. Since this El gamal message is build on a reduction of DDH, the adversary can decide DDH with advantage greater than $\epsilon$ \Lightning.
\\ \\
No such adversary $Adv$ can exist, and El Gamal must be CPA secure.
\end{proof}

\end{document}
