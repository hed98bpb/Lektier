\documentclass[paper=a4, fontsize=11pt]{scrartcl} % A4 paper and 11pt font size
\usepackage[utf8]{inputenc}

\usepackage[T1]{fontenc} % Use 8-bit encoding that has 256 glyphs
\usepackage{fourier} % Use the Adobe Utopia font for the document - comment this line to return to the LaTeX default
\usepackage[english]{babel} % English language/hyphenation
\usepackage{amsmath,amsfonts,amsthm} % Math packages
\usepackage{amssymb}

\usepackage{subcaption}
\usepackage{graphicx}
\usepackage{lipsum} % Used for inserting dummy 'Lorem ipsum' text into the template

\usepackage{sectsty} % Allows customizing section commands
\allsectionsfont{\centering \normalfont\scshape} % Make all sections centered, the default font and small caps

\usepackage{fancyhdr} % Custom headers and footers
\pagestyle{fancyplain} % Makes all pages in the document conform to the custom headers and footers
\fancyhead{} % No page header - if you want one, create it in the same way as the footers below
\fancyfoot[L]{} % Empty left footer
\fancyfoot[C]{} % Empty center footer
\fancyfoot[R]{\thepage} % Page numbering for right footer
\renewcommand{\headrulewidth}{0pt} % Remove header underlines
\renewcommand{\footrulewidth}{0pt} % Remove footer underlines
\setlength{\headheight}{13.6pt} % Customize the height of the header

\numberwithin{equation}{section} % Number equations within sections (i.e. 1.1, 1.2, 2.1, 2.2 instead of 1, 2, 3, 4)
\numberwithin{figure}{section} % Number figures within sections (i.e. 1.1, 1.2, 2.1, 2.2 instead of 1, 2, 3, 4)
\numberwithin{table}{section} % Number tables within sections (i.e. 1.1, 1.2, 2.1, 2.2 instead of 1, 2, 3, 4)

\setlength\parindent{0pt} % Removes all indentation from paragraphs - comment this line for an assignment with lots of text

%----------------------------------------------------------------------------------------
%	TITLE SECTION
%----------------------------------------------------------------------------------------


%----------------------------------------------------------------------------------------
%	TITLE SECTION
%----------------------------------------------------------------------------------------

\newcommand{\horrule}[1]{\rule{\linewidth}{#1}} % Create horizontal rule command with 1 argument of height

\title{	
	\normalfont \normalsize 
	\textsc{Aarhus Universitet, Science, Computer Science} \\ [25pt] % Your university, school and/or department name(s)
	\horrule{0.5pt} \\[0.4cm] % Thin top horizontal rule
	\huge Cryptography - Handin 6 \\ % The assignment title
	\horrule{2pt} \\[0.5cm] % Thick bottom horizontal rule
}

\author{Peter Burgaard - 201209175} % Your name

\date{\normalsize\today} % Today's date or a custom date

\begin{document}

\maketitle

\section{Stinson 5.10}

Prove that:
\begin{align*}
	 x\in \mathbb{Z}_n^*\implies d(e(x))=x
\end{align*}
We know that $e(x)=x^b\mod{n}$, and $d(y)=y^a\mod{n}$, from the exercise description. We know from Stinson, chapter 5, $n=pq$, where $p$ and $q$ are primes where $p\neq q$, and that $ab\equiv1\mod{(p-1)(q-1)}$. The exercise also mentions as a hint that $x_1\equiv x_2\mod{p\cdot q} \iff x_1\equiv x_2\mod{p}$ and $x_1\equiv x_2\mod{q}$ which follows from Stonson, Theorem 5.3. \\

Our goal is to show
\begin{align*}
	d(y)=d(e(x))\equiv \left( x^b\right) ^a\equiv x \mod{n}
\end{align*}
\begin{proof}
	\hspace{1cm} \\ \\
We will approach this by dividing $x$ into an expression with an 'inner' exponent and an 'outer' exponent. Lets make p the 'inner'. By Stinson 5.3
\begin{align*}
	\left( x^b\right) ^a\equiv x^{t\phi(n)+1}\mod{n}\equiv x^{t(p-1)(q-1)+1}\equiv x\cdot \left( x^{(p-1)}\right) ^{t(q-1)}
\end{align*}
for some integer $t\geq1$. Its obvious that if $p*q|x\implies p|x$. So, since $p$ is assumed to be a prime, we can derive from Fermat little theorem\footnote{Suppose $p$ is a prime and $a\in\mathbb{Z}$ where $p\nmid a$, then $a^{p-1}\equiv 1\mod{p}$}:
\begin{align*}
	x\cdot\left( x^{(p-1)}\right) ^{t(q-1)}\equiv x\cdot 1^{t(q-1)}\mod{p} \implies \left( x^b\right) ^a\equiv x \mod{p}
\end{align*}
The same proof can be done for  p as a divider, and by the clue in the exercise we get
\begin{align*}
\left( x^b\right) ^a\equiv x \mod{p\cdot q} \iff
\begin{cases}
\left( x^b\right) ^a\equiv x \mod{p} \\
\left( x^b\right) ^a\equiv x \mod{q} 
\end{cases} 
\end{align*}
And since $p\cdot q=n$, then we are done.
\end{proof}
\section{Stinson 5.10 - continued}
Prove:
\begin{align*}
	x \in \mathbb{Z}_n \implies d(e(x))=x
\end{align*}
\begin{proof}
	\hspace{1cm}\\\\
	Since we made no assumptions about $x\in \mathbb{Z}_n^*$, the proof above must also hold for $x\in\mathbb{Z}_n$
\end{proof}
\end{document}